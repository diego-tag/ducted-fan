\section{TATTU LiPo. Batteria ricaricabile}
Motori, Servomotori e E.S.C richiedono specifici livelli di tensione per il loro funzionamento, tensioni che non possono essere fornite dalla NUCLEO H745ZI-Q.
Si rende, quindi, necessario l'inserimento nel progetto di batterie ricaricabili.\\
La selezione delle batterie disponibili sul mercato è stata condotta a seguito di un'analisi approfondita delle caratteristiche elettriche di motori, servomotori ed E.S.C, 
unitamente ad uno studio dettagliato delle esigenze di autonomia necessarie per le prove di volo, rapportando il tutto al \textit{budget} disponibile.\\
Le \textbf{TATTU LiPo} possiedono le prestazioni richieste dal progetto.

\subsection{TATTU LiPo. Caratteristiche fisiche.}
\begin{itemize}
    \item Dimensioni: $74\times 35.5\times 29$ [mm].
    \item Peso: 155 [g].
    \item Numero di celle: 4.
\end{itemize}

\subsection{TATTU LiPo. Caratteristiche tecniche e considerazione operative.}
\paragraph{\small{LiPo}}\mbox{}\\
La sigla \textit{LiPo}, sta ad indicare la tecnologia elettrochimica utilizzata nella costruzione della batteria. \textit{LiPo} nasce da \textit{Lithium-Polymer}, ovvero polimero di litio. 
La particolare nominazione deriva dal componente chimico utilizzato come elttrolita, che, nel caso in esame, non è liquido, ma un polimero gelificato. Malgrado nelle fonti ufficiali non venga
specificato il polimero usato nel progetto delle \textit{TATTU LiPo},con il fine di informazione, è stato deciso di riportare quello più comunemente utilizzato per la tipologia di batterie in esame, il
\textit{polivinildenfluoro esafluoropropilene} (PVDF-HFP).\\
Rispetto ad altri sistemi elettrochimici presenti sul mercato (come ad esempio: Li-ion, $LiFePo_4$, NiMH ecc..), le batterie LiPo presentano un'elevata densità di potenza, una bassa resistenza interna e una
notevole flessibilità nelle geometrie costruttive, oltre ad un contenuto peso. Tali attributi risultano vantaggiosi nell'ambito della progettazione di UAV, come il \textit{D.P.D.F}.

\paragraph{{\small{Tensione}}}\mbox{}\\
La \textit{TATTU LiPo} possiede \textbf{4} celle da \textbf{3.7 [V]}. Le celle sono collegate in serie, perciò la tensione nominale è circa \textbf{14.8 [V]}.\\
A carica completa, ogni cella può raggiungere i \textbf{4.2 [V]}, emettendo un'uscita di \textbf{16.8 [V]}.
Il mantenimento della tensione delle celle nell'intorno dei 3.7 [V], contribuisce a preservare la stabilità chimica interna delle batterie,
riducendone l'erosione delle celle e prolungandone la vita utile. Si consiglia di non mantenere le celle a 4.2 [V] per periodi prolungati di tempo, evitandone così il danneggiamento.\\
Nella fase iniziale e di sviluppo \textit{firmware}, le batterie sono state tenute in bilancimento alla tensione nominale di 14.8 [V], mentre, 
nella fase finale, in cui i \textit{test} di volo sono stati protagonisti, al fine di ottenere un'aumento dell'autonomia delle batterie e della prestazione dei motori, queste venivano caricate, mantenendo il bilanciamento tra celle, fino a 4.2 [V] per cella.
Infine, è necessario adottare un'ulteriore accortezza per evitare danni. È consigliabile non far scendere la tensione della batteria al di sotto dei \textbf{12 [V]}, mantendo così il valore di tensione di ogni cella sopra i \textbf{3 [V]}.

\paragraph{\small{Capacità}}\mbox{}\\
La \textit{capacità} di una batteria, misurata in [mAh], \textit{milliAmpere-ora}, definisce la quantità di carica elettrica che la batteira può immagazzinare e fornire nel tempo. 
Nel caso in esame, la capacità è di \textbf{1300 [mAh]}, vale a dire che la batteria può fornire \textbf{1.3 [A]} per un'ora. In generale si può utilizzare la seguente formula approssimata per il calcolo del tempo di funzionamento:
\begin{equation}
    t_h = \frac{C\,\,[mAh]}{A\,\,[mA]}
\end{equation} 
dove $t_h$ rappresenta il tempo di funzionamento in ore, $C$ la capacità della batteria espressa in milliAmpere-ora e $A$ la corrente assorbita espressa in milliAmpere.
Con il fine di aumentare l'autonomia di volo, si è optato per l'accoppiamento in parallelo di \textbf{due} \textit{TATTU LiPo}, così da ottenere una capacità totale di \textbf{2600 [mAh]}.
L'accoppiamento in sicurezza è stato effettuato utilizzando il \textit{\textbf{PowerSafe Twin della MNR-electronics}}, il quale verrà approfondito nella successiva sezione.

\paragraph{\small{Capacità di scarica}}\mbox{}\\
Con il termine \textit{capacità di scarica} ci si riferisce alla quantità di corrente massima che la batteria può fornire, in sicurezza e in maniera continuativa, senza danneggiarsi.
La \textit{TATTU LiPo} possiede una capacità di scarica pari a \textbf{75C}, vale a dire che può erogare:
\begin{equation}
    A_{max} = C_s\cdot C \implies A_{max} = 75\,\,[C]\cdot 1.3\,\,[A]= 97.5\,\,[A]
\end{equation}
senza surriscaldarsi o danneggiarsi.\\
 Nella precedente formula, $A_{max}$ si riferisce alla corrente massima erogabile dalla batteria in sicurezza, espressa in [A], \textit{Ampère}. $C_s$ e $C$, invece, si riferiscono rispettivamente alla capacità
di scarica e alla capacità di erogazione di corrente in'ora, espressa sempre in [A].
\begin{figure}[h]
    \centering
    \includegraphics[width=0.6\textwidth]{chapters/Chapter2-HARDWARE/Figures/T.png}
    \caption{TATTU LiPo 4S 1300mAh 75C}
    \label{fig:Hardware drone}
\end{figure}

%% RIFERIMENTO : https://www.dronezine.it/50082/powersafe-twin-avanced-sistema-ridondanza-le-batterie-dei-droni/
\section{PowerSafe Twin ADV. MNR-electronics}
Per garantire la sicurezza e l'affidabilità dell'alimentazione a due pacchi batteria, si è deciso per l'implementazione del
\textbf{modulo di ridondanza, PowerSafe Twin ADV di MNR-electronics}.\\
Il dispositivo incarna un'avanzata soluzione di \textit{ridondanza attiva}. Quest'ultima fa riferimento alla capacità del sistema di intervenire attivamente nel funzionamento del circuito con il fine di garantire continuità di alimentaizone.
Il modulo, di fatti, monitora continuamente alcune variabili di stato, come tensione e corrente, selezionando, in tempo reale, il pacco alimentante il carico. Per esempio, se dovesse abassarsi la tensione di una batteria sotto 
una certa soglia critica, il dispositivo commuterebbe automaticamente sull'altra, evitando così, una significativa interruzione della potenza fornita.\\
Nell'architettura del sistema è possibile distinguere tre moduli distinti.

\paragraph{\small{Modulo di potenza}}\mbox{}\\
Il modulo di potenza è il circuito elettrico responsabile: della commuttazione tra i pacchi batteria, del bilanciamento e della distribuzione di corrente verso il carico.\\
Inoltre, il dispositivo implementa circuiti di protezione, isolando la sorgente guasta, qualora si dovesse presentare un malfunzionamento, ed evitando il trasferimento di corrente fra i pacchi batteria.

\paragraph{\small{Modulo di controllo}}\mbox{}\\
Il modulo di controllo rappresenta il "centro di comando" del dispositivo. L'implementazione di un microprocessore permette la continua misura di parametri decisionali fondamentali, quali: la tensione delle batterie, la corrente di carico
e l'eventuale squilibrio. L'inserimento del modulo di controllo nell'architettura circuitale, permette l'implementazione della, precedentemente citata, tecnologia di \textit{ridondanza attiva}.

\paragraph{\small{Modulo di allarme}}\mbox{}\\
Il modulo di allarme provvede alla segnalazione di eventuali guasti mediante emissioni luminose (LED) ed acustiche (Buzzer).
\begin{figure}[h]
    \centering
    \includegraphics[width=1\textwidth]{chapters/Chapter2-HARDWARE/Figures/Schema collegamenti PowerSafeTwinADVMNRele.png}
    \caption{PowerSafe Twin ADV MNR-electronics. Schema dei collegamenti}
    \label{fig:Hardware drone}
\end{figure}
\subsection{PowerSafe Twin ADV. Caratteristiche fisiche e tecniche}
A seguire, un breve elenco delle caratteristiche fisiche e tecniche del dispositivo:
\begin{itemize}
\item Dimensioni con involucro protettivo: $80\times 59\times 26$ [mm].
\item Peso, considernado sia l'involucro protettivo che i cablaggi (connettori XT60): 100 [g].
\item Corrente massima gestibile: 100 [A] (in continua).
\item Numero di celle per pacco batteria gestibili: $[3S-8S]$
\item Tensione massima ammessa in ingresso: $\approx 33.6$ [V]. Considerando il minimo 3.7 [V] e il massimo 4.2 [V] per cella: $[11,1\div 33,6]$ [V].
\end{itemize}
\begin{figure}[h]
    \centering
    \includegraphics[width=0.6\textwidth]{chapters/Chapter2-HARDWARE/Figures/POWERSAFWMOD.png}
    \caption{PowerSafe Twin ADV MNR-electronics}
    \label{fig:Hardware drone}
\end{figure}

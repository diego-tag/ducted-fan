

\section{DFRobot Power Module. Convertirore di tensione}
Ai fini progettuali sono stati implementati diversi dispositivi, ciascuno dei quali richiedente di una specifica tensione di alimentazione.
Si è reso necessario ricorrere a convertitori di tensione per distribuire nella maniera corretta le tensioni di alimentazione ai vari dispositivi.\\
Il \textbf{DFRobot Power Module}, riceve in ingresso la tensione proveniente dal pacco batterie, erogando quattro diversi valori di tensione a seconda del terminale che si sceglie di utilizzare.\\
La prima soluzione offerta dal dispositivo, fornisce in uscita la stessa tensione che si ha in ingresso, diventando così, un ponte per il passaggio della tensione.\\
Un'altra soluzione consiste nell'utilizzo degli appositi \textit{pin} sul dispositivo, dai quali può essere erogata una tensione di \textbf{5 [V]}, selezionabile mediante un apposito \textit{switch}.\\
Infine, il dispositivo presenta un terminale che eroga una tensione variabile il cui valore è controllato da un \textit{dimmer}. Il \textit{dimmer} permette di selezionare valori intermedi fino ad un massimo
pari alla tensione di ingresso. Quest'ultima è stata utilizzata nell'alimentazione dei servomotori.
La presenza di due servomotori ha richiesto l'integrazione di due convertitori di tensione.

\begin{figure}[htbp]
    \centering
    \includegraphics[width=0.5\textwidth]{chapters/Chapter2-HARDWARE/Figures/Convertitore_di_tensione_foto_tecnica.png}
    \caption{DFRobot Power Module. Disposizione dei terminali di uscita}.
    \label{fig:etichetta}
\end{figure}

\paragraph{Caratteristiche fisiche}
\begin{itemize}
    \item Dimensioni: $46\times 50\times 20$ [mm].
    \item Peso: 35 [g].
\end{itemize}

\paragraph{Caratteristiche tecniche}
\begin{itemize}
    \item Intervallo di tensione in ingresso: \textbf{[3.6$\div$25] [V]}.
    \item Intervallo di tensione esprimibile in uscita: \textbf{[3.3$\div$25] [V]}.
    \item Massima potenza erogabile: \textbf{25 [W]}.
    \item Frequenza di variazione: \textbf{350 [kHz]}.
\end{itemize}

\begin{figure}[htbp]
    \centering
    \includegraphics[width=0.6\textwidth]{chapters/Chapter2-HARDWARE/Figures/Convertitore_Tensione_foto_reale.jpg}
    \caption{DFRobot Power Module}.
    \label{fig:etichetta}
\end{figure}

\section{Power Distribution Board}
Con \textit{Power Distribution Board} si fa riferimento ad un sottosistema elettrico la cui principale funzione è la \textbf{distribuzione in modo sicuro ed efficiente dell'energia fornita dal pacco batterie} agli \textit{E.S.C} e, conseguentemente, ai motori.
Ai fini progettuali, è stata selezionata una \textit{P.D.B}, il cui punto di prelievo del segnale presenta un connettore \textbf{XT60}, adatto al pacco batterie.

\begin{figure}[htbp]
    \centering
    \includegraphics[width=0.6\textwidth]{chapters/Chapter2-HARDWARE/Figures/PowerDistributionBoardQuad.jpg}
    \caption{Power Distribution Board}.
    \label{fig:etichetta}
\end{figure}

Il \textit{P.B.C} integrato nel contesto operativo, possiede quattro uscite eroganti al massimo una corrente di \textbf{20 [A]} ed un peso di: \textbf{27.3 [g]}.
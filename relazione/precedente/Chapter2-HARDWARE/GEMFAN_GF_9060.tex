\section{Eliche}
Al fine di ottenere una corretta realizzazione progettuale, si rende necessario un minuzioso studio circa le eliche da utilizzare.
Le eliche e i motori, costituiscono un perfetto connubbio, di fatti, se si dovessero scegliere le eliche sbagliate, potrebbe non verificarsi la generazione della portanza necessaria per il sollevamento del \textit{D.P.D.F}.\\
Per la selezione, sono stati presi in considerazione i due parametri fondamentali usati nella caratterizzazione delle eliche, il \textbf{diametro} e il \textbf{passo}.\\
Con il primo si indica il diametro della circonferenza che l'elica produce quando è in rotazione. Con il secondo, invece, si fa riferimento alla \textit{distanza teorica} che l'elica percorrerebbe lungo il suo asse
di rotazione in una singola rotazione completa, assumendo che si muova in un fluido perfettamente solido e senza alcun slittamento. Per capire al meglio questo concetto, si suggerisce di pensare ad una vite che compie un giro completo mentre sta venendo inserita nel legno.\\
Il \textbf{diametro} di un'elica ne influenza:
\begin{itemize}
    \item \textbf{Spinta}. Un aumento del diametro comporta un aumento della spinta a parità di RPM.
    \item \textbf{Efficienza}. Un aumento del diametro suggerisce un, seppur leggero, aumento dell'efficienza, dal momento che viene mosso un maggiore volume di aria con meno energia.
    \item \textbf{Coppia del motore richiesta}. Maggiore è il diametro dell'elica, maggiore è la coppia richiesta dai motori per muoverle.
    \item \textbf{Reattività}. Più è alto il valore del diametro e più è alta l'inerzia rotazionale, diminuendo la reattività del sistema.
\end{itemize}
Un elevato diametro è consigliato per sistemi che hanno l'obiettivo di stabilità.
Per quanto riguarda il \textbf{passo}, il parametro ha prestigio su:
\begin{itemize}
    \item \textbf{Velocità in volo orizzontale}. Un aumento del passo costituisce un aumento della velocità espressa orizzonatalmente.
    \item \textbf{Spinta}. La spinta espressa dall'elica aumenta se ne si aumenta il passo, tuttavia il diametro dell'elica ne ha più influeza.
    \item \textbf{Coppia del motore richiesta}. Un maggior passo richiede una maggior coppia indotta dai motori, con conseguente aumento dello stress meccanico.
    \item \textbf{Bassa efficienza di Hovering}. Un aumento del passo comporta un aumento della velocità dell'aria mossa dall'elica, con conseguente diminuzione della stabilità. 
\end{itemize}
Al contrario, un elevato passo è tupico dei sistemi focalizzati sulla velocità orizzontale e verticale.

\subsection{GEMFAN GF 9060. Caratteristiche fisiche}
La denominazione "\textbf{9060}" fa riferimento ai due parametri fondamentali dell'elica, il diametro e il passo, espressi in pollici
\begin{itemize}
    \item Diametro: 9 pollici che sono all'incirca 228,9 [mm].
    \item Passo: 6 pollici, all'incirca 152,4 [mm].
\end{itemize}

%%%%%%%%%%%%%%%%%%%%%%%%%%%%%%%%%%%%%%%%%%%%%%%%%%%%%%%%%%%%%%%%%%%
% TO DO LIST SPIEGAZIONE DEL BOSH SENSORTEC 
% Presentazione del sensore nell'ambito del progetto 
% Presentazione del sensore generale  [OK]
% Principio di misura del sensore: accelerometro, giroscopio e magnetometro [OK]
% Dati tecnici: Dimensioni, peso, alimentaizione, collegamenti, comunicazione 
% Modalità operative del sensore spingendo sulla sensor fusion 

\section{BNO-055 Bosch Sensortec. Unità di misura inerziale e magnetometro}
Al fine di controllare l'orientamento nello spazio del \textit{D.P.D.F} mediante servomotori, si necessita dell'informazione di rotazione del sistema nello spazio di riferimento. Si richiede, dunque, un dispositivo
capace di ottenere tale informazione. Il \textbf{BNO-055} della casa \textbf{Bosch Sensortec}, con la sua accuratezza, è il miglior canditato per tale responsabilità.
Il \textbf{BNO-055} è un sensore di orientamento assoluto a 9 assi sviluppato da \textbf{Bosch Sensortec}.
Il dispositivo è un \textit{System in Package}\textcolor{blue}{[A]} che integra un accelerometro triassiale a 14 bit, un giroscopio triassiale a 16 bit, un sensore geomagnetico triassiale e un
microcontrollore Cortex $M0^+$ a 32 bit incaricato nell'eseguire il \textit{software di sensor fusion} integrato.\\
Il \textit{software}, poc'anzi citato, combina i dati provenienti da accelerometro, giroscopio e magnetometro, fornendo: \textbf{Quaternioni, Angoli di Eulero come pitch, roll e yaw e Vettori di orientamento lineari e gravitazionali}.
Nel presente progetto, il \textit{chip} BNO-055 è stato acquistato con montaggio su \textit{scheda di breakout} incluso, dunque nel paragrafo sottostante sono state aggiunte sia le dimensioni/peso del modulo che le dimensioni/peso del chip.

\subsection{BNO-055 Bosch Sensortec. Catatteristiche fisiche}
\begin{itemize}
    \item dimensioni del modulo: $20\times24\times2$ mm.
    \item dimensioni del chip: $3.8\times5.2\times1.1$ mm.
    \item massa del modulo: 3 g.
    \item massa del chip: $\approx 150 mg$.
\end{itemize}

\subsection{BNO-055 Bosch Sensortec. Infrastruttura di alimentazione, comunicazione ed integrazione del senore}
\paragraph{\small{Alimentazione}}\mbox{}\\
Il sensore ha due distinti ingressi di alimentazione: Il \textbf{$V_{DD}$} e il \textbf{$V_{DDIO}$}.\\
Il $V_{DD}$ è il principale terminale di alimentazione dei sensori. Il valore di tensione iniettabile nell'ingresso in analasi appartiene al seguente intervallo: \textbf{[2.4$\div$3.6] [V]}.\\
Il \textbf{$V_{DDIO}$}, invece, è il distinto ingresso di alimentazione del \textit{$\mu C$} e delle interfacce digitali. In questo caso, il valore di tensione può assumere valori di questo intervallo: \textbf{[1.7$\div$3.6] [V]}.\\
Tuttavia, il dispositivo è montato su di una \textit{breakout board} che offre a disposizione solamente un terminale di alimentazione, chiamato $V_{IN}$, il quale fornisce, mediante un micro regolatore di tensione, l'alimentazione sia al $V_{DD}$ che al $V_{DDIO}$.
Il valore di tensione in ingresso al terminale $V_{IN}$ appartiene all'intervallo: \textbf{[2.4$\div$3.6] [V]}.
Il BNO-055, inoltre, supporta tre differenti modalità di alimentazione: \textit{Normal mode}, \textit{Low power mode} and \textit{Suspend mode}. Per lo sviluppo è stata scelta la \textbf{\textit{Normal mode}}.\\

\paragraph{\small{Protocollo di comunicazione}}\mbox{}\\
Il \textit{BNO-055} offre la possibilità di utilizzare due distinti protocolli di comunicazione seriale, \textit{$I^{2}C$} e \textit{UART}. Nel presente progetto è stato utilizzato esclusivamente il protocollo \textit{$I^{2}C$}.
Come per il \textit{VL53L1X di STMicroelectronics}, precedentemente trattato, il sensore dispone della capacità di comunicare in \textit{Fast Mode}, scambiando byte a 400 [kHz] (oltre che in \textit{Standard Mode a 100 [kHz]}).
Anche in questo caso, sia la \textit{breakout board} che il sensore forniscono i terminali di \textbf{Serial DAta (SDA)} e \textbf{Serial CLock (SCL)}.
\begin{figure}[h!]
    \centering
    \includegraphics[width=0.4\textwidth]{chapters/Chapter2-HARDWARE/Figures/sensors_BNO055_STEMMA_pinouts_top.jpg}
    \caption{\textcolor{black}{BNO-055 Bosch Sensortec. Configuraizone dei pin e breakout board}}
    \label{fig:etichetta}
\end{figure}



\subsection{Principio di misura dell'accelerometro, giroscopio e magnetometro}
Prima di descrivere le \textit{features} dell'accelerometro, giroscopio e magnetometro installati nel \textit{BNO-055}, si ritiene necessario dedicare una sezione di approfondimento circa i principi di funzionamento dei sensori precedentemente citati.
\paragraph{Accelerometro}\mbox{}\\
L'accelerometro integrato nel BNO-055 Bosch Sensortec è di tipo capacitivo M.E.M.S, \textit{Micro-Eletro-Mechanical System}. Il dispositivo consente la misura dell'accelerazione lungo i tre assi cartesiani.\\
Il principio di funzionamento si basa sulla rilevazione delle variazioni di capacità tra microstrutture mobili e fisse realizzate su un substrato di silicio.\\
L'elemento sensibile di ciascun asse è costituito da una massa sospesa, \textit{proof mass}, vincolata da microtravi elastiche ad una cornice ancorata al substrato. In condizioni di quiete, la massa è in equilibrio e la capacità tra le piastre interdigitate\textcolor{blue}{[A]} rimangono simmetriche.\\
Quando il dispositivo è sottoposto ad un'accelerazione lungo uno degli assi sensibili, la massa inerziale si sposta in direzione opposta a quella dell'accelerazione, generando una variazione differenziale delle capacità tra le piastre.\\
La variazione di capacità viene rilevata da un circuito integrato, che la converte in un segnale elettrico, proporzionale alla velocità applicata.\\
Il segnale analogico prodotto, viene successivamente digitalizzato da un convertirore analogico digitale integrato nel chip.

\paragraph{Giroscopio}\mbox{}\\
Il giroscopio integrato nel dispositivo fa parte della categoria M.E.M.S di tipo vibrante, \textit{vibrating structure gyroscope}, e consente la misura della velocità angolare lungo i tre assi cartesiani.\\
Il principio di funzionamento si basa sull'effetto Coriolis, che si manifesta quando una massa in moto oscillatorio subisce una rotazione rispetto ad un sistema di riferimento inerziale.
All'interno del sensore, ciascun asse dispone di una o più masse vibranti, le quali vengono mantenute in oscillazione a frequenza costante, mediante un circuito di attuazione elettrostatica. Quando il dispositivo ruota attorno ad uno degli assi
la massa subisce una forza di Coriolis data da:
\begin{equation}
    \vec{F_c}=2m(\vec{v}\times \vec{\omega }) 
\end{equation}
dove $m$ è la massa oscillante, $\vec{v}$ è la velocità della massa nella sua traiettoria vibrante, $\vec{\omega}$ è la velocità angolare del corpo.\\
Questa forza induce uno spostamento trasversale rispetto alla direzione di vibrazione, che viene rilevato attraverso variazioni di capacità tra elettrodi fissi e mobili, similmente a quanto avviene nell'accelerometro.\\
Tali variazioni, proporzionali alla velocità angolare, vengono convertite in un segnale elettrico mediante un circuito di lettura differenziale e successivamente digitalizzate tramite un convertitore analogico digitale integrato.

\paragraph{Magnetometro}\mbox{}\\
Il sensore integra inoltre un \textbf{magnetometro ad effetto Hall}, sfruttante il fenomeno fisico della tensione di Hall per misurare l'orientamento del dispositivo rispetto al campo magnetico terrestre.\\
Nella sua forma più semplice, un elemento di Hall è una sottilissima lamina di materiale conduttore o semiconduttore attraversata da una corrente continua controllata. Se su questa lamina agisce un campo magnetico con componente perpendicolare alla direzione della corrente,
i portatori di carica vengono deviati lateralmente della \textbf{forza di Lorentz}:
\begin{equation}
    F = q(\vec{v}\times\vec{B})
\end{equation}
con $\vec{v}$ vettore velocità della carica e $\vec{B}$ vettore campo magnetico. Questa deviazione procude un accumulo di carica ai bordi opposti della lamina, e, in regime stazionario, un campo elettrico trasversale 
che equilibria la forza magnetica. Il risultato è una differenza di potenziale trasversale, la \textbf{tensione di Hall}: $V_H$ che risulta proporzionale alla componente di campo magnetico normale alla superficie del 
sensore e alla corrente che lo attraversa. Per un elmento \textit{Hall} omogeneo abbiamo:
\begin{equation}
    V_H = \frac{I\vec{B_{\perp}}}{nqs}
\end{equation}
dove $\vec{B_{\perp}}$ è la componente del campo perpendicolare al piano della lamina, $n$ la densità di portatori, $q$ il valore di carica elementare e $s$ lo spessore del \textit{film} attivo.
Nel magnetometro ogni elemento di Hall è disposto in modo da avere la propria normale allineata con uno degli assi cartesiani del sistema di riferimento del sensore. Quando il campo magnetico terrestre
attraversa il dispositivo, ciascun elemento rileva la proiezione del vettore $\vec{B}$ lungo il proprio asse, trasformandola in una tensione di Hall proporzionale.\\
Dopo un'opportuna amplificazione della tensione di Hall percepita e una conversione analogico-digitale si ottiene un tripletto di valori numerici che rappresentano le tre componenti cartesiane del campo magnetico locale.\\
Il vettore tridimensionale ottenuto viene ricostruito e poi confrontato con il valore atteso del campo geomagnetico.\\
Attraverso il \textit{sensor fusion}, il magnetometro fornisce quindi il riferimento assoluto per l'\textit{azimut}, vale a dire, la direzione del Nord magnetico rispetto al sistema di riferimento solidale del sensore.

\subsection{BNO-055 Bosch Sensortec. Accelerometro, giroscopio e magnetometro integrati}
L'accelerometro, il giroscopio e il magnetometro sono tutti prodotti da Bosch Sensortec.

\paragraph{Accelerometro}\mbox{}\\
L'accelerometro installato sul \textit{BNO-055} appartiene alla famiglia \textit{Bosch Sensortec}. La sua \textbf{risoluzione} è di \textbf{14 bit}.
L'accelerometro, per adattarsi ad una vasta gamma di esigenze progettuali, mette a disposizione diversi intervalli di misura:
\begin{itemize}
    \item $[\pm 2]$ [g]. Modalità utile nella rilevazione di piccolissime accelerazioni.
    \item $[\pm 4]$ [g]. Rappresenta un compromesso equilibrato. Adatto alla magior parte della applicazioni, di fatti, quando attiva la \textit{sensor fusion}, viene impostato tale intervallo di misura.
    \item $[\pm 8]$ [g] e [$\pm 16$] [g]. Le modalità vengono utilizzate in applicazioni dinamiche estremamente veloci. Questi intervalli difficilmente vengono saturati. Nel presente progetto non sono stati nemmeno presi in considerazioni.
\end{itemize}
L'accelerometro, con attiva la \textit{sensor fusion}, ha una banda passante di \textbf{62.5 Hz}.

\paragraph{Giroscopio}\mbox{}\\
Come per l'accelerometro, anche il giroscopio appartiene alla casa \textit{Bosch Sensortec}. La sua \textbf{risoluzine}, però, è di \textbf{16 bit}.
Anche in questo caso si dispone di diversi intervalli di misura:
\begin{itemize}
    \item $[\pm 125]$\degree/s. Modalità pensata per movimenti lenti e precisi, di fatti, possono essere rilevate minime variazioni d'angolo.
    \item $[\pm 250]$\degree/s. Viene conservata la precisione a fronte di un intervallo più permissivo rispetto al precedente.
    \item $[\pm 500]$\degree/s. Questa modalità è adottata dalla maggioranza delle applicazioni.
    \item $[\pm 1000]$\degree/s. La modalità è stata pensata per azionionamenti energici.
    \item $[\pm 2000]$\degree/s. Quest'ampio intervallo è adatto ad applicazioni estremamente veloci. I dati in questa modalità sono estremamente rumorosi. Nonostante questo, è la modalità operativa impostata, quando la \textit{sensor fusion} è attiva. L'estremo rumore di questa modalità, viene attuanuato dalla \textit{sensor fusion}.
\end{itemize}
Ad un'aumento dell'ampiezza dell'intervallo, corrisponde una diminuzione della qualità del dato.
Il giroscopio, nel mentre che la \textit{sensor fusion} è attiva,  ha una banda passante di \textbf{32 Hz}.

\paragraph{Magnetometro}\mbox{}\\
Anche il magnetometro porta il marchio \textit{Bosh Sensortec}. Il dispositivo misura campi nell'ordine dei \textbf{$\pm$ 1300\,\,$[\mu T]$} sugli assi X e Y, e \textbf{$\pm$ 2500\,\,$[\mu T]$} sull'asse Z. Seguendo il precedente ordinamento degli 
assi, la \textbf{risoluzione} è rispettivamente di \textbf{13 bit} e \textbf{15 bit}. La documentazione suggerisce un'elevata sensibilità ai disturbi (\textit{soft iron}, \textit{hard iron}). Difetto che può essere corretto attivando la \textit{sensor fusion}.\\
La banda passante del dispositivo, quando è attivata la\textit{sensor fusion} è di \textbf{10 Hz}.

\paragraph{Sensor fusion}\mbox{}\\
Se presi singolarmente, nessuno dei sensori trattati fornisce misure accurate e complete. Il progetto necessita di un sensore che possa fornire la rotazione del sistema intorno agli assi. 
Questa rotazione è ottenuta dall'integrazione della grandezza estratta dal giroscopio, tuttavia, è intrinsecamente suscettibile ad errori (amplificazione di rumore o \textit{bias}).\\
L'accelerometro fornisce un'informazione assoluta circa la direzione del vettore gravità, da cui possono essere ricavati gli angoli di \textit{roll} e \textit{pitch}. Tuttavia, il limite principale 
risiede nella presenza di accelerazioni non dovute alla gravità, di fatti, rapidi movimenti o vibrazioni producono rumore che interferiscono con la determinazione del vettore gravità.\\
Infine, il magnetometro è suscettibile alla presenza di elementi ferromagnetici o correnti elettriche, determinando, così, errori significativi nella misura.\\
La \textit{sensor fusion} nasce con l'obiettivo di superare tali limiti, combinado coerentemente le informazioni disponibili. Nei sistemi più evoluti, utilizzati nelle piattaforme di navigazione di veicoli
e velivoli, si impiegano filtri come \textit{EKF} (Filtro di Kalman esteso) o \textit{UKF} (Filtro di Kalman a Punti Sigma). La documentazione ufficiale del BNO-055 riporta che il microcontrollore presenta un
algoritmo di \textit{sensor fusion} utilizzate il \textbf{Filtro di Kalman esteso}.\\
Nel presente progetto, con il fine di aumentare l'attendibilità, l'accuratezza e la stabilità della misura, la \textbf{\textit{sensor fusion} è stata sempre tenuta attiva}.\\
A seguire una rappresentazione schematica del \textit{System in package BNO-055 Bosch Sensortec}.
\begin{figure}[h!]
    \centering
    \includegraphics[width=0.4\textwidth]{chapters/Chapter2-HARDWARE/Figures/IMG3-Architettura BNO-055.png}
    \caption{\textcolor{black}{BNO-055 Bosch Sensortec. Architettura di sistema}}
    \label{fig:etichetta}
\end{figure}

\subsection{BNO-055 Bosch Sensortec. Modalità operative}
Il dispositivo di misura fornisce una grande varietà di segnali di \textit{output}, che possono essere scelti selezionando l'appropiata modalità operativa.\\
Le modalità operative vengono classificate in base all'attivazione o meno del \textit{software di sensor fusion}, nello specifico distinguiamo tra le \textit{Non-Fusion modes} e le \textit{Fusion modes}.
\paragraph{Fusion-modes}\mbox{}\\
Nel momento in cui viene impostata una delle seguenti modalità, l'algoritmo di fusione \textbf{esegue automaticamente, in \textit{background}, la calibrazione dei sensori}.
\begin{itemize}
    \item \textbf{IMU (Inertial Measurement Unit)}. In questa modalità di fusione, l'orientamento relativo del \textit{BNO-055} nello spazio, è determinato utilizzado i dati estratti dall'accelerometro e dal giroscopio. La computazione è veloce, di fatti è suggerita nelle applicazioni ad alta velocità.
    \item \textbf{M4G (Magnet for Gyroscope)}. Simile alla precedente modalità, tuttavia, il giroscopio non viene più utilizzato per la rilevazione della rotazione bensì la variazione dell'orientamento del magnetometro nel campo magnetico terrestre.
    \item \textbf{COMPASS}. In questa modalità, il \textit{BNO-055} si comporta come una bussola, determinando l'orientamento del sensore rispetto al polo nord magnetico.
    \item \textbf{NDOF}. In questa modalità, la computazione dell'orientamento del dispositivo, è eseguita servendosi di tutti i dati provenienti dai sensori. La presente modalità è quella con il più alto consumo di corrente.
\end{itemize}
Nel presente progetto, la modalità \textbf{NDOF} è sempre stata scelta come modalità operativa del \textit{BNO-055}.
A seguire, un'immagine riportata dalla documentazione ufficiale, specificante la frequenza con cui il sensore produce dati in uscita, nel mentre che la \textit{sensor fusion} è attiva.
\begin{figure}[h!]
    \centering
    \includegraphics[width=1.0\textwidth]{chapters/Chapter2-HARDWARE/Figures/Frequenza di aggiornamento del BNO-055.png}
    \caption{\textcolor{black}{BNO-055 Bosch Sensortec. Architettura di sistema}}
    \label{fig:etichetta}
\end{figure}
\paragraph{Non-Fusion Modes}
\begin{itemize}
    \item \textbf{ACCONLY}. Il dispositivo fornisce solamente dati grezzi di provenienti dall'accelerometro, ponendo magnetometro e giroscopio nella modalità a basso consumo.
    \item \textbf{MAGONLY}. Se impostata, il dispositivo fornisce solamente i dati del magnetometro, sospendendo l'accelerometro e il magnetometro.
    \item \textbf{GYROONLY}. La modalità è simile alle precedenti, però, in questo caso il dispositivo fornisce solo le misure del giroscopio, sospendendo l'accelerometro ed il magnetometro.
    \item \textbf{ACCMAG}. In questa modalità, il sistema fornisce sia le misure dell'accelerometro che quelle del magnetometro. Il giroscopio è sospeso.
    \item \textbf{ACCGYRO}. Simile alla precedente, ma il sensore rende disponibili solo le misure dell'accelerometro e del giroscopio.
    \item \textbf{MAGGYRO}. In questo caso, il sensore fornisce le misure del magnetometro e del giroscopio, sospendendo l'accelerometro.
    \item \textbf{AMG (ACC-MAG-GYRO)}. Con la presente modalità, il $\mu C$ attiva tutti i sensori, fornendo le misure di accelerometro, giroscopio e magnetometro. 
\end{itemize}



